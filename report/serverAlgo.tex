\section{Algorithme du serveur et structures de données utilisées}
\label{algoServ}

\subsection{Structures de données utilisées}
Pour stocker les informations sur le serveur, nous utilisons une hashmap constituée à l'intérieur de listes chaînées pour gérer les collisions. La hashmap utilise comme 
clef un hashage sur les mots-clefs associés à un fichier.

Lorsqu'un pair envoie un fichier au serveur, pour chaque mot-clef de recherche, on réalise le hashage et on range 
les métadonnées avec l'adresse IP du pair dans la liste chainée correspondant à l'entrée de la hashmap. On utilise pour cela une structure appelée \texttt{Torrent} qui contient 
les métadonnées envoyées par le client et l'adresse IP du pair obtenue via le code.


\subsection{Algorithmes}
\noindent \textbf{Algorithme général du serveur :}

\begin{enumerate}
	\item Ouverture de la connexion UDP sur \texttt{PORT\_ON\_SERV}
	\item Attente de connexion
	\item Si on reçoit une requête \texttt{PUBLISH} : exécution de l'algorithme de publication
	\item Si on reçoit une requête \texttt{SEARCH} : exécution de l'algorithme de recherche
\end{enumerate}

\noindent \textbf{Algorithme de publication}

\begin{enumerate}
	\item Réponse \texttt{PUBLISH\_READY} au client depuis un autre port généré par le système (pour ne pas gêner l'écoute UDP)
	\item Réception des métadonnées du client
	\item Envoi de l'acquittement \texttt{PUBLISH\_ACK}
	\item Pour chaque mot-clef de recherche des métadonnées :
		\begin{enumerate}
			\item Hash du mot-clef
			\item Ajout des métadonnées et de l'adresse IP du pair (sous la forme d'une structure \texttt{Torrent}) à la liste contenue dans l'entrée de la hashmap correspondant au hash du mot-clef
		\end{enumerate}
\end{enumerate}


\noindent \textbf{Algorithme de recherche}

\begin{enumerate}
	\item Réponse \texttt{SEARCH\_READY} au client depuis un autre port généré par le système (pour ne pas gêner l'écoute UDP)
	\item Réception du mot-clef de recherche du client
	\item Hash de mot-clef
	\item Parcours de la liste contenue à \texttt{hashmap[hash\_keyword]} : pour chaque entrée, si le mot-clef est contenu dans les tags des métadonnées, incrémenter le compteur de résultats
	\item Envoi du nombre de résultats au client
	\item Si le nombre de résultats est différent de 0 :
		\begin{enumerate}
			\item Se repositionner sur la liste à \texttt{hasmap[hash\_keyword]}
			\item La parcourir en envoyant au client chaque entrée où le mot-clef est contenu dans les tags de la métadonnée.
		\end{enumerate}
\end{enumerate}