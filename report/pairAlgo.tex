\section{Algorithme d'un pair et structures de données utilisées}

\subsection{Structures de données utilisées}

Sur un pair, nous avons besoin de stocker les fichiers publiés par ce pair sur le serveur. Pour cela, nous utilisons une hashmap contenant des listes chaînées pour gérer 
les collisions sur une même clef. Cette hashmap contient des éléments typés par la structure \texttt{LocalFile}. Celle-ci contient les métadonnées du fichier (structure
\texttt{Metadata}) ainsi que son emplacement sur le disque dur (sous la forme d'une chaîne de caractères).

\subsection{Algorithmes}

\noindent \textit{Remarque :} l'ensemble des phases d'attente sont soumises à un timeout. Quand celui-ci expire, un message est affiché à l'utilisateur l'invitant à recommencer l'opération lancée.

\noindent \textbf{Algorithme général}

\begin{enumerate}
	\item Création d'un thread servant d'écoute TCP (voir algorithme de partage)
	\item Si l'utilisateur demande à envoyer un fichier, exécution de l'algorithme de publication
	\item Si l'utilisateur demande à effectuer une recherche, exécution de l'algorithme de recherche
\end{enumerate}

\noindent \textbf{Algorithme de partage}

\begin{enumerate}
	\item Création d'une socket d'écoute sur le port \texttt{PORT\_ON\_CLIENT}. Celle-ci est ajoutée à un \texttt{select} ce qui nous permet de gérer plusieurs téléchargements simultanés
	\item Lors d'une demande de connexion d'un client : on parcourt la liste des fichiers publiés et on récupère la structure de type \texttt{LocalFile} dont le SHA correspond 
	à celui envoyé par le pair distant.
	\item On crée une nouvelle de socket de dialogue pour le client, qu'on écoutera via l'instruction \texttt{select}
	\item Sur cette socket, on commence à envoyer le fichier au fur et à mesure en fonction des limites du protocole TCP
	\item Lorsque l'envoi est terminé, on ferme la socket et on la retire du \texttt{select}
\end{enumerate}

\noindent \textbf{Algorithme de publication}

\begin{enumerate}
	\item Récupération auprès de l'utilisateur du chemin du fichier à envoyer
	\item Vérification que le fichier existe
	\item Construction des métadonnées (extension récupérée automatiquement, calcul du SHA sur le contenu du fichier, mots-clefs récupérés par une saisie utilisateur)
	\item Envoi d'une demande de publication \texttt{PUBLISH} au serveur
	\item Attente d'une réponse \texttt{PUBLISH\_READY}
	\item Envoi des métadonnées du fichier au serveur sur le nouveau port d'origine de \texttt{PUBLISH\_READY}
	\item Attente de l'acquittement \texttt{PUBLISH\_ACK}
	\item Retour à l'utilisateur
\end{enumerate}

\noindent \textbf{Algorithme de recherche et de récupération de fichier}

\begin{enumerate}
	\item Récupération par saisie utilisateur du mot-clef de recherche
	\item Envoi d'une requête \texttt{SEARCH}
	\item Attente de réception de la réponse \texttt{SEARCH\_READY}
	\item Envoi du mot-clef de recherche au serveur sur le port d'origine de la réponse \texttt{SEARCH\_READY}
	\item Attente de réception du nombre de résultats trouvés
	\item Si celui-ci est différent de 0 :
		\begin{enumerate}
			\item Attente de réception de chaque résultat
			\item S'il manque un résultat, afficher un message d'erreur et inviter l'utilisateur à recommencer sa requête
		\end{enumerate}
	\item Si l'utilisateur choisit de télécharger un fichier en réponse de cette recherche, nous envoyons un \texttt{GET sha(fichier)} au pair dont l'adresse 
	est contenue dans le résultat de la requête. Le téléchargement se met alors en marche et lorsqu'on reçoit le signal de fin de fichier, nous fermons 
	la connexion.
\end{enumerate}
